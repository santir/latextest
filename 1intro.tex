\section{Introducción}
\label{sec:intro}


El presente trabajo se basa en el estudio de diversos tipos de circuitos que se pueden lograr utilizando amplificadores operacionales, buscando entender su funcionamiento, comportamiento ideal y, tras haber realizado las correspondientes mediciones en el laboratorio, concluir respecto de cuales son las limitaciones de ese modelo. Además, se analizan dos tipos de circuitos rectificadores de media onda, estudiando cómo varía su funcionamiento al modificar la carga, aumentar la frecuencia, entre otros.

{\bf El barrido en frecuencia de la figura 2.14 creo que convendría hacerlo en dB y el eje de la frecuencia logarítmico. De esa manera va a ser más fácil identificar el polo.}

\todo[inline, color=green!40]{Listo ahi lo arregle - Santiago.}

{\bf Che, estoy pensando en cómo justificar la respuesta del filtro logarítmico... y la verdad, no los entiendo del todo los gráficos de Vo/vi... son las señales que parten de las imágenes 39, 40 y 41. Si les llega a agarrar algo de interés, podrían chequearlo? - Osea, el gráfico hay que verlo al revés porque invierte, y se le ve "un poco la forma del logaritmo", pero no estoy seguro de eso...}

\todo[inline, color=green!40]{No se si cambiaste algo desde que hiciste este comentario pero no hay imágenes 39,40 y 41 jaja. Respecto de las figuras que hacen Vo vs Vi lo acabo de ver y tampoco lo entiendo, si se me ocurre algo aviso. - Santiago.

jaja, me estaba refiriendo a las imagenes que sacamos del osciloscopio SCD0000039,40,41. ahí escribí algo... pero creo que es chamuyo..}

\todo[inline, color=red!40]{che, hay algo de la Fig. \ref{fig:multipl_a_idealvsmed} que no me cierra.
Contra qué están comparando el desfasaje? Contra la señal de entrada? Yo no veo que la fig 2.5 esté desfasada.
%Porque lo que entiendo yo de ese punto es que el desfasaje no lo provocaría el AO, sino que la carga capacitiva del osciloscopio.
Y en la fig 2.6, no es que hay desfasaje, están a distinta frecuencia nomás. La medida está a una frecuencia apenas un poco menor a 1kHz - No quiero tocar por si me equivoco, pero fíjense en eso. - Lucho}


\todo[inline, color=red!40]{En la pregunta de "aumentar a más de 0.4V y ver qué pasa" de la parte del inversor no me acuerdo qué pasaba. Recortaba la señal? atenuaba? Podrás extender un toque la respuesta?}



\todo[inline, color=red!40]{En la caption de las figuras de barrido de frecuencia del inversor, convendría poner si es de las mediciones o de simulación. cómo es? la primera es simulación x10 y la segunda es medición x1? (figuras 2.15 y 2.17}


\todo[inline, color=red!40]{Entonces... hay que hacer la medición de corriente para el rectificador agregándole una resistencia pequeña

corregir lo que decís del gráfico de esas "senoidales desfasadas"

corregir la respuesta a la última pregunta del inversor, (agregar una foto del análisis de fourier quizás), y explicar el fenómeno de la deformación de la senoidal al transformarla en triangular.}

\todo[inline, color=green!40]{Ahi puse lo del método para medir corriente que circula por el diodo, por si acaso que alguno le pegue una leída.

También corregí lo de las figuras de barrido en frecuencia que no aclaraban que correspondían a una simulación}


