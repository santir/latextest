\section{Conclusiones}
\label{sec:conclu}

Durante el desarrollo de este trabajo se fueron verificando las hipótesis con las que se partió: los modelos ideales son aplicables solamente en determinadas situaciones. En el caso de no tener en cuenta estas condiciones se pueden encontrar con resultados imprevistos a la hora de trabajar con circuitos cuyo correcto funcionamiento solo puede ser predicho si se trabaja con una modelización apropiada.

En principio, bajo las condiciones que se analizaron, se encontró que el amplificador operacional tiene un ancho de banda limitado, esto implica que su impedancia de salida no es cero. También, en la experiencia en la que se trabajó con resistencias del orden de $1M\Omega$, se corroboró que la impedancia de entrada no es infinito. A su vez, su salida no puede superar la tensión de alimentación, lo que produce las alinealidades y los recortes ya vistos.

En el caso de los circuitos rectificadores ésto se evidenció principalmente a la hora de intentar aplicar al diodo señales de alta frecuencia para las cuales no fue diseñado, lo que introdujo un efecto capacitivo al sistema.

Por otro lado, en algunos ejercicios también fue necesario considerar los modelos no ideales de los instrumentos para que la respuesta de la simulación sea semejante a la medición. Esto sucedió en la deformación de la señal generada en el rectificador.

% Habías escrito algo del ancho de banda del operacional antes, lo corregí porque el BW del opamp es de 1MHz, y eso no lo probamos.

